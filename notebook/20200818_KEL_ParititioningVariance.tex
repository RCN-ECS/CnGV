\documentclass[]{article}
\usepackage{lmodern}
\usepackage{amssymb,amsmath}
\usepackage{ifxetex,ifluatex}
\usepackage{fixltx2e} % provides \textsubscript
\ifnum 0\ifxetex 1\fi\ifluatex 1\fi=0 % if pdftex
  \usepackage[T1]{fontenc}
  \usepackage[utf8]{inputenc}
\else % if luatex or xelatex
  \ifxetex
    \usepackage{mathspec}
  \else
    \usepackage{fontspec}
  \fi
  \defaultfontfeatures{Ligatures=TeX,Scale=MatchLowercase}
\fi
% use upquote if available, for straight quotes in verbatim environments
\IfFileExists{upquote.sty}{\usepackage{upquote}}{}
% use microtype if available
\IfFileExists{microtype.sty}{%
\usepackage{microtype}
\UseMicrotypeSet[protrusion]{basicmath} % disable protrusion for tt fonts
}{}
\usepackage[margin=1in]{geometry}
\usepackage{hyperref}
\hypersetup{unicode=true,
            pdftitle={New way of creating and standardizing phenotype data with variance partitioning},
            pdfauthor={KE Lotterhos},
            pdfborder={0 0 0},
            breaklinks=true}
\urlstyle{same}  % don't use monospace font for urls
\usepackage{color}
\usepackage{fancyvrb}
\newcommand{\VerbBar}{|}
\newcommand{\VERB}{\Verb[commandchars=\\\{\}]}
\DefineVerbatimEnvironment{Highlighting}{Verbatim}{commandchars=\\\{\}}
% Add ',fontsize=\small' for more characters per line
\usepackage{framed}
\definecolor{shadecolor}{RGB}{248,248,248}
\newenvironment{Shaded}{\begin{snugshade}}{\end{snugshade}}
\newcommand{\AlertTok}[1]{\textcolor[rgb]{0.94,0.16,0.16}{#1}}
\newcommand{\AnnotationTok}[1]{\textcolor[rgb]{0.56,0.35,0.01}{\textbf{\textit{#1}}}}
\newcommand{\AttributeTok}[1]{\textcolor[rgb]{0.77,0.63,0.00}{#1}}
\newcommand{\BaseNTok}[1]{\textcolor[rgb]{0.00,0.00,0.81}{#1}}
\newcommand{\BuiltInTok}[1]{#1}
\newcommand{\CharTok}[1]{\textcolor[rgb]{0.31,0.60,0.02}{#1}}
\newcommand{\CommentTok}[1]{\textcolor[rgb]{0.56,0.35,0.01}{\textit{#1}}}
\newcommand{\CommentVarTok}[1]{\textcolor[rgb]{0.56,0.35,0.01}{\textbf{\textit{#1}}}}
\newcommand{\ConstantTok}[1]{\textcolor[rgb]{0.00,0.00,0.00}{#1}}
\newcommand{\ControlFlowTok}[1]{\textcolor[rgb]{0.13,0.29,0.53}{\textbf{#1}}}
\newcommand{\DataTypeTok}[1]{\textcolor[rgb]{0.13,0.29,0.53}{#1}}
\newcommand{\DecValTok}[1]{\textcolor[rgb]{0.00,0.00,0.81}{#1}}
\newcommand{\DocumentationTok}[1]{\textcolor[rgb]{0.56,0.35,0.01}{\textbf{\textit{#1}}}}
\newcommand{\ErrorTok}[1]{\textcolor[rgb]{0.64,0.00,0.00}{\textbf{#1}}}
\newcommand{\ExtensionTok}[1]{#1}
\newcommand{\FloatTok}[1]{\textcolor[rgb]{0.00,0.00,0.81}{#1}}
\newcommand{\FunctionTok}[1]{\textcolor[rgb]{0.00,0.00,0.00}{#1}}
\newcommand{\ImportTok}[1]{#1}
\newcommand{\InformationTok}[1]{\textcolor[rgb]{0.56,0.35,0.01}{\textbf{\textit{#1}}}}
\newcommand{\KeywordTok}[1]{\textcolor[rgb]{0.13,0.29,0.53}{\textbf{#1}}}
\newcommand{\NormalTok}[1]{#1}
\newcommand{\OperatorTok}[1]{\textcolor[rgb]{0.81,0.36,0.00}{\textbf{#1}}}
\newcommand{\OtherTok}[1]{\textcolor[rgb]{0.56,0.35,0.01}{#1}}
\newcommand{\PreprocessorTok}[1]{\textcolor[rgb]{0.56,0.35,0.01}{\textit{#1}}}
\newcommand{\RegionMarkerTok}[1]{#1}
\newcommand{\SpecialCharTok}[1]{\textcolor[rgb]{0.00,0.00,0.00}{#1}}
\newcommand{\SpecialStringTok}[1]{\textcolor[rgb]{0.31,0.60,0.02}{#1}}
\newcommand{\StringTok}[1]{\textcolor[rgb]{0.31,0.60,0.02}{#1}}
\newcommand{\VariableTok}[1]{\textcolor[rgb]{0.00,0.00,0.00}{#1}}
\newcommand{\VerbatimStringTok}[1]{\textcolor[rgb]{0.31,0.60,0.02}{#1}}
\newcommand{\WarningTok}[1]{\textcolor[rgb]{0.56,0.35,0.01}{\textbf{\textit{#1}}}}
\usepackage{graphicx}
% grffile has become a legacy package: https://ctan.org/pkg/grffile
\IfFileExists{grffile.sty}{%
\usepackage{grffile}
}{}
\makeatletter
\def\maxwidth{\ifdim\Gin@nat@width>\linewidth\linewidth\else\Gin@nat@width\fi}
\def\maxheight{\ifdim\Gin@nat@height>\textheight\textheight\else\Gin@nat@height\fi}
\makeatother
% Scale images if necessary, so that they will not overflow the page
% margins by default, and it is still possible to overwrite the defaults
% using explicit options in \includegraphics[width, height, ...]{}
\setkeys{Gin}{width=\maxwidth,height=\maxheight,keepaspectratio}
\IfFileExists{parskip.sty}{%
\usepackage{parskip}
}{% else
\setlength{\parindent}{0pt}
\setlength{\parskip}{6pt plus 2pt minus 1pt}
}
\setlength{\emergencystretch}{3em}  % prevent overfull lines
\providecommand{\tightlist}{%
  \setlength{\itemsep}{0pt}\setlength{\parskip}{0pt}}
\setcounter{secnumdepth}{0}
% Redefines (sub)paragraphs to behave more like sections
\ifx\paragraph\undefined\else
\let\oldparagraph\paragraph
\renewcommand{\paragraph}[1]{\oldparagraph{#1}\mbox{}}
\fi
\ifx\subparagraph\undefined\else
\let\oldsubparagraph\subparagraph
\renewcommand{\subparagraph}[1]{\oldsubparagraph{#1}\mbox{}}
\fi

%%% Use protect on footnotes to avoid problems with footnotes in titles
\let\rmarkdownfootnote\footnote%
\def\footnote{\protect\rmarkdownfootnote}

%%% Change title format to be more compact
\usepackage{titling}

% Create subtitle command for use in maketitle
\providecommand{\subtitle}[1]{
  \posttitle{
    \begin{center}\large#1\end{center}
    }
}

\setlength{\droptitle}{-2em}

  \title{New way of creating and standardizing phenotype data with variance
partitioning}
    \pretitle{\vspace{\droptitle}\centering\huge}
  \posttitle{\par}
    \author{KE Lotterhos}
    \preauthor{\centering\large\emph}
  \postauthor{\par}
      \predate{\centering\large\emph}
  \postdate{\par}
    \date{8/18/2020}


\begin{document}
\maketitle

\hypertarget{variance-partitioning}{%
\section{Variance partitioning}\label{variance-partitioning}}

\(V_P = V_G + V_E + V_{GE} + Cov_{GE}\)

In a reciprocal transplant experiment, there are \(g\) genotypes
transplanted into \(e\) environmental patches, for a total of
\(g*e = n_{GE}\) genotype-environment combinations.

\(g\) is levels of genotypes for \(i = 1,2... g\) \(e\) is levels of
environments for \(j = 1,2... e\)

Assuming the equal sample sizes \(r\) (\(1, 2, ...k\)) within each
genotype-environment combination, could partitioning the variance be as
simple as:

\(V_G = re\sum_{i=1}^g (\bar{y_i} - \bar{y})^2\)

\(V_E = rg\sum_{j=1}^e (\bar{y_j} - \bar{y})^2\)

\(V_{GE} = r \sum_{i=1}^g \sum_{j=1}^e (\bar{y_{ij}} - \bar{y_i} - \bar{y_j} + \bar{y})^2\)

\(V_{Cov_{GE}}= \frac{ger}{(\sum_{i=1}^g\sum_{j=1}^eI_{ij})} \sum_{i=1}^g\sum_{j=1}^e(\bar{y_i} - \bar{y})(\bar{y_j} - \bar{y})I\)

\begin{itemize}
\tightlist
\item
  I is an indicator variable that is 1 when the genotype \(i\)
  originated from environment \(j\) and 0 otherwise
\item
  basically, each summation needs to count the same number of times to
  be comparable as a SS on the same scale.
\item
  If we have a 2x2 reciprocal transplant, we have 2 of 4 treatment
  combinations that count toward the sum (and 2 that do not). If we have
  a 4x4 reciprocal transplat, we have 4 of 16 reciprocal transplants
  that count toward the sum (and 12 that do not). If we have 8 genotypes
  in 2 environments (4 from each env), then we have 4 genotypes that
  count toward the cov and (4 that do not). The ones that that count
  make this SS not
\item
  We can get this SS to a comparable amount to the other SS by
  multiplying by the factor at the beginning of the equation.
\end{itemize}

\(V_{error} = \sum_{i=1}^g \sum_{j=1}^e \sum_{k=1}^r (y_{ijk}-\bar{y}_{ij})^2\)

Some evidence that this is behaving as expected:

\begin{itemize}
\item
  if you set beta1 = 1 and beta2 = 1 and int = 0, with little error, you
  get equal amounts of variance explained for G, E, and CovGE for n pops
\item
  if you set beta1 = 1 and beta2 = 0.1 and int = 0, you get more
  variance explained for V\_E, a little bit for V\_G, and a decent
  amount for cov\_GE
\item
  if you set beta1 = 0.1 and beta2 = 1 and int = 0, you get the same
  pattern as above but V\_E and V\_G switched
\item
  if you set beta1 or beta2 negative and the other positive, you get
  negative covGE
\item
  try adding noise and see if outputs make sense

  \begin{itemize}
  \tightlist
  \item
    Adding noise can sometimes ``create'' some CovGE when there are
    enough pops and beta1=0 or beta2=0
  \item
    But for most part seems to work (?)
  \end{itemize}
\end{itemize}

\hypertarget{effect-size}{%
\section{Effect size}\label{effect-size}}

\(Cov_{GE}= \frac{1}{\sum_{i=1}^g\sum_{j=1}^gI-1}\frac{\sum_{i=1}^g\sum_{j=1}^e(\bar{y}_i} - \bar{y})(\bar{y}_j} - \bar{y})}{max(\sigma_{\bar{y}_i}},\sigma_{\bar{y}_j})}I\)

\hypertarget{step-1-create-true-phenotype-data-without-error}{%
\section{Step 1: create true phenotype data without
error}\label{step-1-create-true-phenotype-data-without-error}}

\begin{Shaded}
\begin{Highlighting}[]
\NormalTok{beta1 =}\StringTok{ }\DecValTok{-1} \CommentTok{# the amount the phenotype changes across 1 value of the environment (i.e., the slope). This is essentially the amount/degree of phenotypic plasticity that is the same across genotypes.}

\NormalTok{n_genotypes =}\StringTok{ }\DecValTok{10} \CommentTok{# the number of genotypes collected from different locations}
\NormalTok{n =}\StringTok{ }\DecValTok{10} \CommentTok{# sample size}
\NormalTok{beta2 =}\StringTok{ }\DecValTok{1} \CommentTok{# the amount the phenotype changes from one genotype to the next. This is essitially the increase intercept from one genotype to the next.}

\NormalTok{scale =}\StringTok{ }\FloatTok{0.01} \CommentTok{# the scale for sd_noise within pops compared to sd_means among pops}
\CommentTok{# I'll let you explore the space, but I think the upper would be scale = 1}
\CommentTok{# I'd start with scale = 1 and scale = 1/2 and see what that looks like}

\NormalTok{sd_int =}\StringTok{ }\DecValTok{0}\CommentTok{#n_genotypes # sd of the interaction terms that will be drawn}
    \CommentTok{#  beta1 <- 1}
    \CommentTok{#  beta2 <- 1}
    
    \CommentTok{# Generate data}
    
\NormalTok{    n_environments =}\StringTok{ }\NormalTok{n_genotypes }\CommentTok{# the number of common garden environments that the genotypes are grown in. Basically, genotype[i] orginates from environment[i]. This matching between genotypes and environment is important for the the cov(G,E) calculation. Here, we only consider the case of the complete reciprocal transplant experiment.}

    \CommentTok{## Create levels of genotypes and environments ####}
\NormalTok{    G =}\StringTok{ }\KeywordTok{rep}\NormalTok{(}\DecValTok{1}\OperatorTok{:}\NormalTok{n_genotypes, }\DataTypeTok{each=}\NormalTok{n}\OperatorTok{*}\NormalTok{n_genotypes) }\CommentTok{# '0' n times, '1' n times}
\NormalTok{    E =}\StringTok{ }\KeywordTok{rep}\NormalTok{(}\DecValTok{1}\OperatorTok{:}\NormalTok{n_environments,}\DataTypeTok{times=}\NormalTok{n}\OperatorTok{*}\NormalTok{n_genotypes) }\CommentTok{# '0'x50, '1'x50, '0'x50, '1'x50}
    \CommentTok{#set.seed(97)}
    

    \CommentTok{## Create interaction effect for each level of both factors ####}
    \CommentTok{# In this case we assume the GxE interactions are a }
    \CommentTok{# normally distributed random variable with a mean of 0}
    \CommentTok{# As the sd increases, so does the GxE among treatments}
    \ControlFlowTok{if}\NormalTok{(sd_int }\OperatorTok{==}\StringTok{ }\DecValTok{0}\NormalTok{)\{}
\NormalTok{      int =}\StringTok{ }\DecValTok{0}
\NormalTok{    \}}\ControlFlowTok{else}\NormalTok{\{}
\NormalTok{    int <-}\StringTok{ }\KeywordTok{rnorm}\NormalTok{(n_genotypes }\OperatorTok{*}\StringTok{ }\NormalTok{n_environments, }\DecValTok{0}\NormalTok{, }\DataTypeTok{sd=}\NormalTok{sd_int)}
\NormalTok{    \}}
    \CommentTok{# no GxE}
 
     \CommentTok{# this sd determines the amount of GxE}
\NormalTok{    int_df <-}\StringTok{ }\KeywordTok{data.frame}\NormalTok{(}\KeywordTok{expand.grid}\NormalTok{(}\DataTypeTok{G=}\DecValTok{1}\OperatorTok{:}\NormalTok{n_genotypes, }\DataTypeTok{E=}\DecValTok{1}\OperatorTok{:}\NormalTok{n_environments), }
\NormalTok{                         int)}
    
    \CommentTok{### Create the model dataframe ####}
\NormalTok{    model_df <-}\StringTok{ }\KeywordTok{data.frame}\NormalTok{(G, E)}
\NormalTok{    model_df}\OperatorTok{$}\NormalTok{GE_factor <-}\StringTok{ }\KeywordTok{paste0}\NormalTok{(}\StringTok{"G"}\NormalTok{,model_df}\OperatorTok{$}\NormalTok{G, }\StringTok{"E"}\NormalTok{, model_df}\OperatorTok{$}\NormalTok{E)}
\NormalTok{    model_df <-}\StringTok{ }\KeywordTok{merge}\NormalTok{(model_df, int_df)}
\NormalTok{    model_df <-}\StringTok{ }\NormalTok{model_df[}\KeywordTok{order}\NormalTok{(model_df}\OperatorTok{$}\NormalTok{G, model_df}\OperatorTok{$}\NormalTok{E),]}
    \KeywordTok{dim}\NormalTok{(model_df)}
\end{Highlighting}
\end{Shaded}

\begin{verbatim}
## [1] 1000    4
\end{verbatim}

\begin{Shaded}
\begin{Highlighting}[]
    \KeywordTok{head}\NormalTok{(model_df, }\DecValTok{30}\NormalTok{)}
\end{Highlighting}
\end{Shaded}

\begin{verbatim}
##    G E GE_factor int
## 1  1 1      G1E1   0
## 2  1 1      G1E1   0
## 3  1 1      G1E1   0
## 4  1 1      G1E1   0
## 5  1 1      G1E1   0
## 6  1 1      G1E1   0
## 7  1 1      G1E1   0
## 8  1 1      G1E1   0
## 9  1 1      G1E1   0
## 10 1 1      G1E1   0
## 21 1 2      G1E2   0
## 22 1 2      G1E2   0
## 23 1 2      G1E2   0
## 24 1 2      G1E2   0
## 25 1 2      G1E2   0
## 26 1 2      G1E2   0
## 27 1 2      G1E2   0
## 28 1 2      G1E2   0
## 29 1 2      G1E2   0
## 30 1 2      G1E2   0
## 31 1 3      G1E3   0
## 32 1 3      G1E3   0
## 33 1 3      G1E3   0
## 34 1 3      G1E3   0
## 35 1 3      G1E3   0
## 36 1 3      G1E3   0
## 37 1 3      G1E3   0
## 38 1 3      G1E3   0
## 39 1 3      G1E3   0
## 40 1 3      G1E3   0
\end{verbatim}

\begin{Shaded}
\begin{Highlighting}[]
    \CommentTok{# Generate phenotype data using the regression equation ####}
\NormalTok{    model_df}\OperatorTok{$}\NormalTok{GE_true =}\StringTok{ }\NormalTok{beta1}\OperatorTok{*}\NormalTok{model_df}\OperatorTok{$}\NormalTok{E }\OperatorTok{+}\StringTok{ }\NormalTok{beta2}\OperatorTok{*}\NormalTok{model_df}\OperatorTok{$}\NormalTok{G }\OperatorTok{+}\StringTok{ }\NormalTok{model_df}\OperatorTok{$}\NormalTok{int}
    
\NormalTok{    G_true <-}\StringTok{ }\KeywordTok{data.frame}\NormalTok{(}\DataTypeTok{G=}\DecValTok{1}\OperatorTok{:}\NormalTok{n_genotypes, }\DataTypeTok{G_true=}\KeywordTok{tapply}\NormalTok{(model_df}\OperatorTok{$}\NormalTok{GE_true, model_df}\OperatorTok{$}\NormalTok{G, mean))}
\NormalTok{    E_true <-}\StringTok{ }\KeywordTok{data.frame}\NormalTok{(}\DataTypeTok{E=}\DecValTok{1}\OperatorTok{:}\NormalTok{n_genotypes , }\DataTypeTok{E_true=}\KeywordTok{tapply}\NormalTok{(model_df}\OperatorTok{$}\NormalTok{GE_true, model_df}\OperatorTok{$}\NormalTok{E, mean))}
    
\NormalTok{    model_df1 <-}\StringTok{ }\KeywordTok{merge}\NormalTok{(model_df,G_true)}
\NormalTok{    model_df2 <-}\StringTok{ }\KeywordTok{merge}\NormalTok{(model_df1,E_true)}
\NormalTok{    model_df <-}\StringTok{ }\NormalTok{model_df2}
    
\NormalTok{    model_df}\OperatorTok{$}\NormalTok{mean_true <-}\StringTok{ }\KeywordTok{mean}\NormalTok{(model_df}\OperatorTok{$}\NormalTok{GE_true)}
    
\NormalTok{    model_df}\OperatorTok{$}\NormalTok{int_true <-}\StringTok{ }\NormalTok{model_df}\OperatorTok{$}\NormalTok{mean_true }\OperatorTok{+}\StringTok{ }\NormalTok{model_df}\OperatorTok{$}\NormalTok{GE_true }\OperatorTok{-}\StringTok{ }
\StringTok{                        }\NormalTok{model_df}\OperatorTok{$}\NormalTok{G_true }\OperatorTok{-}\StringTok{   }\NormalTok{model_df}\OperatorTok{$}\NormalTok{E_true}
    \KeywordTok{head}\NormalTok{(model_df)}
\end{Highlighting}
\end{Shaded}

\begin{verbatim}
##   E G GE_factor int GE_true G_true E_true mean_true int_true
## 1 1 1      G1E1   0       0   -4.5    4.5         0        0
## 2 1 1      G1E1   0       0   -4.5    4.5         0        0
## 3 1 1      G1E1   0       0   -4.5    4.5         0        0
## 4 1 1      G1E1   0       0   -4.5    4.5         0        0
## 5 1 1      G1E1   0       0   -4.5    4.5         0        0
## 6 1 1      G1E1   0       0   -4.5    4.5         0        0
\end{verbatim}

\begin{Shaded}
\begin{Highlighting}[]
    \KeywordTok{tail}\NormalTok{(model_df)}
\end{Highlighting}
\end{Shaded}

\begin{verbatim}
##       E  G GE_factor int GE_true G_true E_true mean_true int_true
## 995  10 10    G10E10   0       0    4.5   -4.5         0        0
## 996  10 10    G10E10   0       0    4.5   -4.5         0        0
## 997  10 10    G10E10   0       0    4.5   -4.5         0        0
## 998  10 10    G10E10   0       0    4.5   -4.5         0        0
## 999  10 10    G10E10   0       0    4.5   -4.5         0        0
## 1000 10 10    G10E10   0       0    4.5   -4.5         0        0
\end{verbatim}

\begin{Shaded}
\begin{Highlighting}[]
    \KeywordTok{par}\NormalTok{(}\DataTypeTok{mfrow=}\KeywordTok{c}\NormalTok{(}\DecValTok{1}\NormalTok{,}\DecValTok{1}\NormalTok{))}
    \KeywordTok{plot}\NormalTok{(model_df}\OperatorTok{$}\NormalTok{int, model_df}\OperatorTok{$}\NormalTok{int_true)}
    \KeywordTok{abline}\NormalTok{(}\DecValTok{0}\NormalTok{,}\DecValTok{1}\NormalTok{)}
\end{Highlighting}
\end{Shaded}

\includegraphics{20200818_KEL_ParititioningVariance_files/figure-latex/unnamed-chunk-1-1.pdf}

Note that the way we add ``int'' to create phenotypic variation is not
the same way that the actual interaction is calculated from the data.
Our simulations create an interaction, but as we create an interaction
we also change the G-means and E-means, and so the calculated
interaction is a bit different.

\hypertarget{step-2-standardize-true-phenotype-data}{%
\section{Step 2: standardize true phenotype
data}\label{step-2-standardize-true-phenotype-data}}

\begin{Shaded}
\begin{Highlighting}[]
\NormalTok{GE_true_means <-}\StringTok{ }\KeywordTok{tapply}\NormalTok{(model_df}\OperatorTok{$}\NormalTok{GE_true, model_df}\OperatorTok{$}\NormalTok{GE_factor, mean)}

\NormalTok{model_df}\OperatorTok{$}\NormalTok{GE_stn_true <-}\StringTok{ }\NormalTok{(model_df}\OperatorTok{$}\NormalTok{GE_true }\OperatorTok{-}\StringTok{ }\KeywordTok{mean}\NormalTok{(GE_true_means))}\OperatorTok{/}\KeywordTok{sd}\NormalTok{(GE_true_means)}

\NormalTok{GE_stn_true_means <-}\StringTok{ }\KeywordTok{tapply}\NormalTok{(model_df}\OperatorTok{$}\NormalTok{GE_stn_true, model_df}\OperatorTok{$}\NormalTok{GE_factor, mean)}

\NormalTok{    G_stn_true <-}\StringTok{ }\KeywordTok{data.frame}\NormalTok{(}\DataTypeTok{G=}\DecValTok{1}\OperatorTok{:}\NormalTok{n_genotypes, }\DataTypeTok{G_stn_true=}\KeywordTok{tapply}\NormalTok{(model_df}\OperatorTok{$}\NormalTok{GE_stn_true, model_df}\OperatorTok{$}\NormalTok{G, mean))}
\NormalTok{    E_stn_true <-}\StringTok{ }\KeywordTok{data.frame}\NormalTok{(}\DataTypeTok{E=}\DecValTok{1}\OperatorTok{:}\NormalTok{n_genotypes , }\DataTypeTok{E_stn_true=}\KeywordTok{tapply}\NormalTok{(model_df}\OperatorTok{$}\NormalTok{GE_stn_true, model_df}\OperatorTok{$}\NormalTok{E, mean))}
    
\NormalTok{    model_df1 <-}\StringTok{ }\KeywordTok{merge}\NormalTok{(model_df,G_stn_true)}
\NormalTok{    model_df2 <-}\StringTok{ }\KeywordTok{merge}\NormalTok{(model_df1,E_stn_true)}
\NormalTok{    model_df <-}\StringTok{ }\NormalTok{model_df2}

\NormalTok{    model_df}\OperatorTok{$}\NormalTok{mean_stn_true <-}\StringTok{ }\KeywordTok{mean}\NormalTok{(model_df}\OperatorTok{$}\NormalTok{GE_stn_true)}
    
\NormalTok{    model_df}\OperatorTok{$}\NormalTok{int_stn_true <-}\StringTok{ }\NormalTok{model_df}\OperatorTok{$}\NormalTok{mean_stn_true }\OperatorTok{+}\StringTok{ }\NormalTok{model_df}\OperatorTok{$}\NormalTok{GE_stn_true }\OperatorTok{-}\StringTok{ }
\StringTok{                        }\NormalTok{model_df}\OperatorTok{$}\NormalTok{G_stn_true }\OperatorTok{-}\StringTok{   }\NormalTok{model_df}\OperatorTok{$}\NormalTok{E_stn_true}
    \KeywordTok{head}\NormalTok{(model_df)}
\end{Highlighting}
\end{Shaded}

\begin{verbatim}
##   E G GE_factor int GE_true G_true E_true mean_true int_true GE_stn_true
## 1 1 1      G1E1   0       0   -4.5    4.5         0        0           0
## 2 1 1      G1E1   0       0   -4.5    4.5         0        0           0
## 3 1 1      G1E1   0       0   -4.5    4.5         0        0           0
## 4 1 1      G1E1   0       0   -4.5    4.5         0        0           0
## 5 1 1      G1E1   0       0   -4.5    4.5         0        0           0
## 6 1 1      G1E1   0       0   -4.5    4.5         0        0           0
##   G_stn_true E_stn_true mean_stn_true int_stn_true
## 1   -1.10227    1.10227             0            0
## 2   -1.10227    1.10227             0            0
## 3   -1.10227    1.10227             0            0
## 4   -1.10227    1.10227             0            0
## 5   -1.10227    1.10227             0            0
## 6   -1.10227    1.10227             0            0
\end{verbatim}

\begin{Shaded}
\begin{Highlighting}[]
    \KeywordTok{tail}\NormalTok{(model_df)}
\end{Highlighting}
\end{Shaded}

\begin{verbatim}
##       E G GE_factor int GE_true G_true E_true mean_true int_true GE_stn_true
## 995  10 9     G9E10   0      -1    3.5   -4.5         0        0  -0.2449490
## 996  10 7     G7E10   0      -3    1.5   -4.5         0        0  -0.7348469
## 997  10 6     G6E10   0      -4    0.5   -4.5         0        0  -0.9797959
## 998  10 3     G3E10   0      -7   -2.5   -4.5         0        0  -1.7146428
## 999  10 8     G8E10   0      -2    2.5   -4.5         0        0  -0.4898979
## 1000 10 6     G6E10   0      -4    0.5   -4.5         0        0  -0.9797959
##      G_stn_true E_stn_true mean_stn_true int_stn_true
## 995   0.8573214   -1.10227             0 0.000000e+00
## 996   0.3674235   -1.10227             0 2.220446e-16
## 997   0.1224745   -1.10227             0 0.000000e+00
## 998  -0.6123724   -1.10227             0 0.000000e+00
## 999   0.6123724   -1.10227             0 0.000000e+00
## 1000  0.1224745   -1.10227             0 0.000000e+00
\end{verbatim}

\begin{Shaded}
\begin{Highlighting}[]
    \KeywordTok{par}\NormalTok{(}\DataTypeTok{mfrow=}\KeywordTok{c}\NormalTok{(}\DecValTok{1}\NormalTok{,}\DecValTok{1}\NormalTok{))}
    \KeywordTok{plot}\NormalTok{(model_df}\OperatorTok{$}\NormalTok{int_true, model_df}\OperatorTok{$}\NormalTok{int_stn_true)}
\end{Highlighting}
\end{Shaded}

\includegraphics{20200818_KEL_ParititioningVariance_files/figure-latex/unnamed-chunk-2-1.pdf}

\begin{Shaded}
\begin{Highlighting}[]
      \CommentTok{# awesome}
\end{Highlighting}
\end{Shaded}

\hypertarget{step-3-add-error-to-stnd-phenotype-data}{%
\section{Step 3: add error to stnd phenotype
data}\label{step-3-add-error-to-stnd-phenotype-data}}

scaled to the variation in the phenotype means

\begin{Shaded}
\begin{Highlighting}[]
\KeywordTok{sd}\NormalTok{(GE_stn_true_means)}
\end{Highlighting}
\end{Shaded}

\begin{verbatim}
## [1] 1
\end{verbatim}

\begin{Shaded}
\begin{Highlighting}[]
\CommentTok{# I guess this will always be 1 because we standardize.}
\NormalTok{sd_noise <-}\StringTok{ }\KeywordTok{sd}\NormalTok{(}\KeywordTok{as.numeric}\NormalTok{(GE_stn_true_means))}\OperatorTok{*}\NormalTok{scale}
\NormalTok{model_df}\OperatorTok{$}\NormalTok{e =}\StringTok{ }\KeywordTok{rnorm}\NormalTok{(n}\OperatorTok{*}\NormalTok{n_genotypes}\OperatorTok{*}\NormalTok{n_environments, }\DecValTok{0}\NormalTok{, }\DataTypeTok{sd=}\NormalTok{sd_noise) }\CommentTok{# Random noise}

\NormalTok{model_df}\OperatorTok{$}\NormalTok{phenotype <-}\StringTok{ }\NormalTok{model_df}\OperatorTok{$}\NormalTok{GE_stn_true }\OperatorTok{+}\StringTok{ }\NormalTok{model_df}\OperatorTok{$}\NormalTok{e}
\KeywordTok{head}\NormalTok{(model_df)}
\end{Highlighting}
\end{Shaded}

\begin{verbatim}
##   E G GE_factor int GE_true G_true E_true mean_true int_true GE_stn_true
## 1 1 1      G1E1   0       0   -4.5    4.5         0        0           0
## 2 1 1      G1E1   0       0   -4.5    4.5         0        0           0
## 3 1 1      G1E1   0       0   -4.5    4.5         0        0           0
## 4 1 1      G1E1   0       0   -4.5    4.5         0        0           0
## 5 1 1      G1E1   0       0   -4.5    4.5         0        0           0
## 6 1 1      G1E1   0       0   -4.5    4.5         0        0           0
##   G_stn_true E_stn_true mean_stn_true int_stn_true             e     phenotype
## 1   -1.10227    1.10227             0            0  0.0034127009  0.0034127009
## 2   -1.10227    1.10227             0            0 -0.0027790453 -0.0027790453
## 3   -1.10227    1.10227             0            0 -0.0017247534 -0.0017247534
## 4   -1.10227    1.10227             0            0 -0.0044903729 -0.0044903729
## 5   -1.10227    1.10227             0            0  0.0070030613  0.0070030613
## 6   -1.10227    1.10227             0            0  0.0009617227  0.0009617227
\end{verbatim}

\hypertarget{step-4-plot-the-pattern-so-we-can-see-what-it-looks-like}{%
\section{Step 4: plot the pattern so we can see what it looks
like}\label{step-4-plot-the-pattern-so-we-can-see-what-it-looks-like}}

\begin{Shaded}
\begin{Highlighting}[]
   \KeywordTok{par}\NormalTok{(}\DataTypeTok{mfrow=}\KeywordTok{c}\NormalTok{(}\DecValTok{1}\NormalTok{,}\DecValTok{1}\NormalTok{), }\DataTypeTok{mar=}\KeywordTok{c}\NormalTok{(}\DecValTok{4}\NormalTok{,}\DecValTok{4}\NormalTok{,}\DecValTok{1}\NormalTok{,}\DecValTok{1}\NormalTok{))}
    \KeywordTok{plot}\NormalTok{(model_df}\OperatorTok{$}\NormalTok{phenotype}\OperatorTok{~}\NormalTok{model_df}\OperatorTok{$}\NormalTok{E, }\DataTypeTok{col=}\DecValTok{0}\NormalTok{, }\DataTypeTok{xlim=}\KeywordTok{c}\NormalTok{(}\DecValTok{1}\NormalTok{,n_genotypes}\OperatorTok{+}\DecValTok{1}\NormalTok{), }\DataTypeTok{ylim=}\KeywordTok{c}\NormalTok{(}\OperatorTok{-}\DecValTok{3}\NormalTok{,}\DecValTok{3}\NormalTok{))}
    \ControlFlowTok{for}\NormalTok{ (m }\ControlFlowTok{in} \DecValTok{1}\OperatorTok{:}\NormalTok{n_genotypes)\{}
\NormalTok{      x <-}\StringTok{ }\NormalTok{model_df}\OperatorTok{$}\NormalTok{E[model_df}\OperatorTok{$}\NormalTok{G}\OperatorTok{==}\NormalTok{m]}\OperatorTok{+}\FloatTok{0.05}\OperatorTok{*}\NormalTok{(m}\DecValTok{-1}\NormalTok{)}
      \KeywordTok{points}\NormalTok{(model_df}\OperatorTok{$}\NormalTok{phenotype[model_df}\OperatorTok{$}\NormalTok{G}\OperatorTok{==}\NormalTok{m]}\OperatorTok{~}\NormalTok{x, }\DataTypeTok{col=}\NormalTok{m}\OperatorTok{+}\DecValTok{1}\NormalTok{)}
\NormalTok{      p_means <-}\StringTok{ }\KeywordTok{tapply}\NormalTok{(model_df}\OperatorTok{$}\NormalTok{phenotype[model_df}\OperatorTok{$}\NormalTok{G}\OperatorTok{==}\NormalTok{m],model_df}\OperatorTok{$}\NormalTok{E[model_df}\OperatorTok{$}\NormalTok{G}\OperatorTok{==}\NormalTok{m], mean)}
      \KeywordTok{points}\NormalTok{(p_means}\OperatorTok{~}\KeywordTok{as.numeric}\NormalTok{(}\KeywordTok{names}\NormalTok{(p_means)),}
             \DataTypeTok{col=}\NormalTok{m}\OperatorTok{+}\DecValTok{1}\NormalTok{, }\DataTypeTok{type=}\StringTok{"l"}\NormalTok{)}
\NormalTok{    \}}
\end{Highlighting}
\end{Shaded}

\includegraphics{20200818_KEL_ParititioningVariance_files/figure-latex/unnamed-chunk-4-1.pdf}

\hypertarget{step-5-pretend-we-start-with-noisy-simulated-data}{%
\section{Step 5: pretend we start with noisy simulated
data}\label{step-5-pretend-we-start-with-noisy-simulated-data}}

Go through the steps we will go through in the study

\begin{Shaded}
\begin{Highlighting}[]
\CommentTok{# Step A: standardize by mean(GE_means) and sd(GE_means)}
\NormalTok{phen_GE_obs_means <-}\StringTok{ }\KeywordTok{tapply}\NormalTok{(model_df}\OperatorTok{$}\NormalTok{phenotype, model_df}\OperatorTok{$}\NormalTok{GE_factor, mean)}

\KeywordTok{plot}\NormalTok{(GE_stn_true_means, phen_GE_obs_means) }\CommentTok{#sanity check}
\KeywordTok{abline}\NormalTok{(}\DecValTok{0}\NormalTok{,}\DecValTok{1}\NormalTok{)}
\end{Highlighting}
\end{Shaded}

\includegraphics{20200818_KEL_ParititioningVariance_files/figure-latex/unnamed-chunk-5-1.pdf}

\begin{Shaded}
\begin{Highlighting}[]
\NormalTok{model_df}\OperatorTok{$}\NormalTok{phenotype_stn <-}\StringTok{ }\NormalTok{(model_df}\OperatorTok{$}\NormalTok{phenotype}\OperatorTok{-}\KeywordTok{mean}\NormalTok{(phen_GE_obs_means))}\OperatorTok{/}\KeywordTok{sd}\NormalTok{(phen_GE_obs_means)}

\KeywordTok{plot}\NormalTok{(model_df}\OperatorTok{$}\NormalTok{phenotype, model_df}\OperatorTok{$}\NormalTok{phenotype_stn)}
\KeywordTok{abline}\NormalTok{(}\DecValTok{0}\NormalTok{,}\DecValTok{1}\NormalTok{)}
\end{Highlighting}
\end{Shaded}

\includegraphics{20200818_KEL_ParititioningVariance_files/figure-latex/unnamed-chunk-5-2.pdf}

\begin{Shaded}
\begin{Highlighting}[]
\CommentTok{# Step B: calculate observed G_means and E_means and interaction}
\NormalTok{    G_stn_est <-}\StringTok{ }\KeywordTok{data.frame}\NormalTok{(}\DataTypeTok{G=}\DecValTok{1}\OperatorTok{:}\NormalTok{n_genotypes, }\DataTypeTok{G_stn_est=}\KeywordTok{tapply}\NormalTok{(model_df}\OperatorTok{$}\NormalTok{phenotype_stn, model_df}\OperatorTok{$}\NormalTok{G, mean))}
\NormalTok{    E_stn_est <-}\StringTok{ }\KeywordTok{data.frame}\NormalTok{(}\DataTypeTok{E=}\DecValTok{1}\OperatorTok{:}\NormalTok{n_genotypes , }\DataTypeTok{E_stn_est=}\KeywordTok{tapply}\NormalTok{(model_df}\OperatorTok{$}\NormalTok{phenotype_stn, model_df}\OperatorTok{$}\NormalTok{E, mean))}
    
\NormalTok{    GE_stn_est_means <-}\StringTok{ }\KeywordTok{tapply}\NormalTok{(model_df}\OperatorTok{$}\NormalTok{phenotype_stn, model_df}\OperatorTok{$}\NormalTok{GE_factor, mean)}
\NormalTok{    GE_stn_est_means_df <-}\StringTok{ }\KeywordTok{data.frame}\NormalTok{(}\DataTypeTok{GE_factor=}\KeywordTok{names}\NormalTok{(GE_stn_est_means) , }\DataTypeTok{GE_stn_est=}\NormalTok{GE_stn_est_means)}
    
\NormalTok{    model_df1 <-}\StringTok{ }\KeywordTok{merge}\NormalTok{(model_df,G_stn_est)}
\NormalTok{    model_df2 <-}\StringTok{ }\KeywordTok{merge}\NormalTok{(model_df1,E_stn_est)}
\NormalTok{    model_df3 <-}\StringTok{ }\KeywordTok{merge}\NormalTok{(model_df2, GE_stn_est_means_df)}
\NormalTok{    model_df <-}\StringTok{ }\NormalTok{model_df3}
    
    \KeywordTok{plot}\NormalTok{(model_df}\OperatorTok{$}\NormalTok{GE_stn_true, model_df}\OperatorTok{$}\NormalTok{GE_stn_est)}
    \KeywordTok{abline}\NormalTok{(}\DecValTok{0}\NormalTok{,}\DecValTok{1}\NormalTok{) }\CommentTok{# looks good}
\end{Highlighting}
\end{Shaded}

\includegraphics{20200818_KEL_ParititioningVariance_files/figure-latex/unnamed-chunk-5-3.pdf}

\begin{Shaded}
\begin{Highlighting}[]
\NormalTok{    model_df}\OperatorTok{$}\NormalTok{mean_stn_est <-}\StringTok{ }\KeywordTok{mean}\NormalTok{(model_df}\OperatorTok{$}\NormalTok{phenotype_stn)}
    
    \CommentTok{# Calculate interaction}
\NormalTok{    model_df}\OperatorTok{$}\NormalTok{int_stn_est <-}\StringTok{ }\NormalTok{model_df}\OperatorTok{$}\NormalTok{mean_stn_est }\OperatorTok{+}\StringTok{ }\NormalTok{model_df}\OperatorTok{$}\NormalTok{GE_stn_est }\OperatorTok{-}\StringTok{ }
\StringTok{                        }\NormalTok{model_df}\OperatorTok{$}\NormalTok{G_stn_est }\OperatorTok{-}\StringTok{   }\NormalTok{model_df}\OperatorTok{$}\NormalTok{E_stn_est}
\end{Highlighting}
\end{Shaded}

\hypertarget{step-6-compare-true-values-to-estimated-values}{%
\section{Step 6: Compare true values to estimated
values}\label{step-6-compare-true-values-to-estimated-values}}

\begin{Shaded}
\begin{Highlighting}[]
\KeywordTok{head}\NormalTok{(model_df)}
\end{Highlighting}
\end{Shaded}

\begin{verbatim}
##   GE_factor E  G int GE_true G_true E_true mean_true int_true GE_stn_true
## 1     G10E1 1 10   0       9    4.5    4.5         0        0    2.204541
## 2     G10E1 1 10   0       9    4.5    4.5         0        0    2.204541
## 3     G10E1 1 10   0       9    4.5    4.5         0        0    2.204541
## 4     G10E1 1 10   0       9    4.5    4.5         0        0    2.204541
## 5     G10E1 1 10   0       9    4.5    4.5         0        0    2.204541
## 6     G10E1 1 10   0       9    4.5    4.5         0        0    2.204541
##   G_stn_true E_stn_true mean_stn_true int_stn_true            e phenotype
## 1    1.10227    1.10227             0            0  0.003444012  2.207985
## 2    1.10227    1.10227             0            0  0.011321137  2.215862
## 3    1.10227    1.10227             0            0 -0.002250266  2.202291
## 4    1.10227    1.10227             0            0 -0.013903696  2.190637
## 5    1.10227    1.10227             0            0  0.010758599  2.215299
## 6    1.10227    1.10227             0            0 -0.002266453  2.202274
##   phenotype_stn G_stn_est E_stn_est GE_stn_est  mean_stn_est  int_stn_est
## 1      2.207582  1.103228  1.102453   2.204166 -7.496512e-18 -0.001514885
## 2      2.215457  1.103228  1.102453   2.204166 -7.496512e-18 -0.001514885
## 3      2.201890  1.103228  1.102453   2.204166 -7.496512e-18 -0.001514885
## 4      2.190240  1.103228  1.102453   2.204166 -7.496512e-18 -0.001514885
## 5      2.214895  1.103228  1.102453   2.204166 -7.496512e-18 -0.001514885
## 6      2.201874  1.103228  1.102453   2.204166 -7.496512e-18 -0.001514885
\end{verbatim}

\begin{Shaded}
\begin{Highlighting}[]
\NormalTok{(true_int <-}\StringTok{ }\KeywordTok{mean}\NormalTok{(}\KeywordTok{abs}\NormalTok{(model_df}\OperatorTok{$}\NormalTok{int_stn_true)))}
\end{Highlighting}
\end{Shaded}

\begin{verbatim}
## [1] 5.440093e-17
\end{verbatim}

\begin{Shaded}
\begin{Highlighting}[]
\NormalTok{(obs_int <-}\StringTok{ }\KeywordTok{mean}\NormalTok{(}\KeywordTok{abs}\NormalTok{(model_df}\OperatorTok{$}\NormalTok{int_stn_est)))}
\end{Highlighting}
\end{Shaded}

\begin{verbatim}
## [1] 0.002487851
\end{verbatim}

\hypertarget{new-step-6.5-make-anova-table-with-covge}{%
\section{New step 6.5: make ANOVA table with
CovGE}\label{new-step-6.5-make-anova-table-with-covge}}

Here, I didn't write loops to calculate the equations, so the code will
look different from the equations.

TO DO: check to code the loops as written in the equations and make sure
it gives the same answer as what I calculated here

\begin{Shaded}
\begin{Highlighting}[]
\KeywordTok{head}\NormalTok{(model_df)}
\end{Highlighting}
\end{Shaded}

\begin{verbatim}
##   GE_factor E  G int GE_true G_true E_true mean_true int_true GE_stn_true
## 1     G10E1 1 10   0       9    4.5    4.5         0        0    2.204541
## 2     G10E1 1 10   0       9    4.5    4.5         0        0    2.204541
## 3     G10E1 1 10   0       9    4.5    4.5         0        0    2.204541
## 4     G10E1 1 10   0       9    4.5    4.5         0        0    2.204541
## 5     G10E1 1 10   0       9    4.5    4.5         0        0    2.204541
## 6     G10E1 1 10   0       9    4.5    4.5         0        0    2.204541
##   G_stn_true E_stn_true mean_stn_true int_stn_true            e phenotype
## 1    1.10227    1.10227             0            0  0.003444012  2.207985
## 2    1.10227    1.10227             0            0  0.011321137  2.215862
## 3    1.10227    1.10227             0            0 -0.002250266  2.202291
## 4    1.10227    1.10227             0            0 -0.013903696  2.190637
## 5    1.10227    1.10227             0            0  0.010758599  2.215299
## 6    1.10227    1.10227             0            0 -0.002266453  2.202274
##   phenotype_stn G_stn_est E_stn_est GE_stn_est  mean_stn_est  int_stn_est
## 1      2.207582  1.103228  1.102453   2.204166 -7.496512e-18 -0.001514885
## 2      2.215457  1.103228  1.102453   2.204166 -7.496512e-18 -0.001514885
## 3      2.201890  1.103228  1.102453   2.204166 -7.496512e-18 -0.001514885
## 4      2.190240  1.103228  1.102453   2.204166 -7.496512e-18 -0.001514885
## 5      2.214895  1.103228  1.102453   2.204166 -7.496512e-18 -0.001514885
## 6      2.201874  1.103228  1.102453   2.204166 -7.496512e-18 -0.001514885
\end{verbatim}

\begin{Shaded}
\begin{Highlighting}[]
\NormalTok{V_G_SS =}\StringTok{ }\KeywordTok{sum}\NormalTok{((model_df}\OperatorTok{$}\NormalTok{G_stn_est}\OperatorTok{-}\NormalTok{model_df}\OperatorTok{$}\NormalTok{mean_stn_est)}\OperatorTok{^}\DecValTok{2}\NormalTok{)}
\NormalTok{V_E_SS =}\StringTok{ }\KeywordTok{sum}\NormalTok{((model_df}\OperatorTok{$}\NormalTok{E_stn_est}\OperatorTok{-}\NormalTok{model_df}\OperatorTok{$}\NormalTok{mean_stn_est)}\OperatorTok{^}\DecValTok{2}\NormalTok{)}
\NormalTok{V_GE_SS =}\StringTok{ }\KeywordTok{sum}\NormalTok{(model_df}\OperatorTok{$}\NormalTok{int_stn_est}\OperatorTok{^}\DecValTok{2}\NormalTok{)}
\NormalTok{V_error =}\StringTok{ }\KeywordTok{sum}\NormalTok{((model_df}\OperatorTok{$}\NormalTok{phenotype_stn }\OperatorTok{-}\StringTok{ }\NormalTok{model_df}\OperatorTok{$}\NormalTok{GE_stn_est)}\OperatorTok{^}\DecValTok{2}\NormalTok{)}

\NormalTok{model_df}\OperatorTok{$}\NormalTok{I =}\StringTok{ }\NormalTok{model_df}\OperatorTok{$}\NormalTok{E}\OperatorTok{==}\NormalTok{model_df}\OperatorTok{$}\NormalTok{G}

\CommentTok{# Covariance pattern (ignore the 0,0 points)}
\KeywordTok{par}\NormalTok{(}\DataTypeTok{mfrow=}\KeywordTok{c}\NormalTok{(}\DecValTok{2}\NormalTok{,}\DecValTok{1}\NormalTok{))}
\KeywordTok{plot}\NormalTok{(model_df}\OperatorTok{$}\NormalTok{G_stn_est}\OperatorTok{*}\NormalTok{model_df}\OperatorTok{$}\NormalTok{I, model_df}\OperatorTok{$}\NormalTok{E_stn_est}\OperatorTok{*}\NormalTok{model_df}\OperatorTok{$}\NormalTok{I)}
\KeywordTok{plot}\NormalTok{(model_df}\OperatorTok{$}\NormalTok{G_stn_est}\OperatorTok{*}\NormalTok{model_df}\OperatorTok{$}\NormalTok{I, model_df}\OperatorTok{$}\NormalTok{E_stn_est}\OperatorTok{*}\NormalTok{model_df}\OperatorTok{$}\NormalTok{I, }\DataTypeTok{xlim=}\KeywordTok{c}\NormalTok{(}\OperatorTok{-}\DecValTok{1}\NormalTok{,}\DecValTok{1}\NormalTok{), }\DataTypeTok{ylim=}\KeywordTok{c}\NormalTok{(}\OperatorTok{-}\DecValTok{1}\NormalTok{,}\DecValTok{1}\NormalTok{))}
\end{Highlighting}
\end{Shaded}

\includegraphics{20200818_KEL_ParititioningVariance_files/figure-latex/unnamed-chunk-7-1.pdf}

\begin{Shaded}
\begin{Highlighting}[]
\NormalTok{V_Cov_GE <-}\StringTok{  }\KeywordTok{nrow}\NormalTok{(model_df)}\OperatorTok{/}\KeywordTok{sum}\NormalTok{(model_df}\OperatorTok{$}\NormalTok{I)}\OperatorTok{*}
\StringTok{  }\KeywordTok{sum}\NormalTok{((model_df}\OperatorTok{$}\NormalTok{G_stn_est}\OperatorTok{-}\NormalTok{model_df}\OperatorTok{$}\NormalTok{mean_stn_est)}\OperatorTok{*}\NormalTok{(model_df}\OperatorTok{$}\NormalTok{E_stn_est}\OperatorTok{-}\NormalTok{model_df}\OperatorTok{$}\NormalTok{mean_stn_est)}\OperatorTok{*}\NormalTok{model_df}\OperatorTok{$}\NormalTok{I)}

\NormalTok{SS <-}\StringTok{ }\KeywordTok{round}\NormalTok{(}\KeywordTok{rbind}\NormalTok{(V_G_SS, V_E_SS, V_GE_SS, V_Cov_GE, V_error),}\DecValTok{2}\NormalTok{)}
\NormalTok{omega2 <-}\StringTok{ }\KeywordTok{round}\NormalTok{(}\KeywordTok{abs}\NormalTok{(SS)}\OperatorTok{/}\KeywordTok{sum}\NormalTok{(}\KeywordTok{abs}\NormalTok{(SS)),}\DecValTok{2}\NormalTok{)}
\KeywordTok{data.frame}\NormalTok{(SS, }\KeywordTok{abs}\NormalTok{(SS), omega2)}
\end{Highlighting}
\end{Shaded}

\begin{verbatim}
##               SS abs.SS. omega2
## V_G_SS    495.27  495.27   0.33
## V_E_SS    494.72  494.72   0.33
## V_GE_SS     0.01    0.01   0.00
## V_Cov_GE -494.99  494.99   0.33
## V_error     0.09    0.09   0.00
\end{verbatim}

\hypertarget{step-7-pretend-we-start-with-noisy-empirical-data}{%
\section{Step 7: pretend we start with noisy empirical
data}\label{step-7-pretend-we-start-with-noisy-empirical-data}}

Let's back transform the noisy data, then go through the steps and see
what happens

\begin{Shaded}
\begin{Highlighting}[]
\CommentTok{# what we did to standardize: model_df$GE_stn_true <- (model_df$GE_true - mean(GE_true_means))/sd(GE_true_means)}
\NormalTok{model_df2 <-}\StringTok{ }\NormalTok{model_df[,}\DecValTok{1}\OperatorTok{:}\DecValTok{3}\NormalTok{]}
\NormalTok{model_df2}\OperatorTok{$}\NormalTok{phenotype2 <-}\StringTok{ }\NormalTok{model_df}\OperatorTok{$}\NormalTok{phenotype}\OperatorTok{*}\KeywordTok{sd}\NormalTok{(GE_true_means)}\OperatorTok{+}\KeywordTok{mean}\NormalTok{(GE_true_means) }\CommentTok{#backtransform}


\CommentTok{# Step A: standardize by mean(GE_means) and sd(GE_means)}
\NormalTok{phen_GE_obs_means <-}\StringTok{ }\KeywordTok{tapply}\NormalTok{(model_df2}\OperatorTok{$}\NormalTok{phenotype2, model_df2}\OperatorTok{$}\NormalTok{GE_factor, mean)}

\NormalTok{model_df2}\OperatorTok{$}\NormalTok{phenotype2_stn <-}\StringTok{ }\NormalTok{(model_df2}\OperatorTok{$}\NormalTok{phenotype2}\OperatorTok{-}\KeywordTok{mean}\NormalTok{(phen_GE_obs_means))}\OperatorTok{/}\KeywordTok{sd}\NormalTok{(phen_GE_obs_means)}

\KeywordTok{plot}\NormalTok{(model_df2}\OperatorTok{$}\NormalTok{phenotype2, model_df2}\OperatorTok{$}\NormalTok{phenotype2_stn) }\CommentTok{# straight line check}
\end{Highlighting}
\end{Shaded}

\includegraphics{20200818_KEL_ParititioningVariance_files/figure-latex/unnamed-chunk-8-1.pdf}

\begin{Shaded}
\begin{Highlighting}[]
\CommentTok{# Step B: calculate observed G_means and E_means and interaction}
\NormalTok{    G_stn_est <-}\StringTok{ }\KeywordTok{data.frame}\NormalTok{(}\DataTypeTok{G=}\DecValTok{1}\OperatorTok{:}\NormalTok{n_genotypes, }\DataTypeTok{G_stn_est=}\KeywordTok{tapply}\NormalTok{(model_df2}\OperatorTok{$}\NormalTok{phenotype2_stn, model_df2}\OperatorTok{$}\NormalTok{G, mean))}
\NormalTok{    E_stn_est <-}\StringTok{ }\KeywordTok{data.frame}\NormalTok{(}\DataTypeTok{E=}\DecValTok{1}\OperatorTok{:}\NormalTok{n_genotypes , }\DataTypeTok{E_stn_est=}\KeywordTok{tapply}\NormalTok{(model_df2}\OperatorTok{$}\NormalTok{phenotype2_stn, model_df2}\OperatorTok{$}\NormalTok{E, mean))}
    
\NormalTok{    GE_stn_est_means <-}\StringTok{ }\KeywordTok{tapply}\NormalTok{(model_df2}\OperatorTok{$}\NormalTok{phenotype2_stn, model_df2}\OperatorTok{$}\NormalTok{GE_factor, mean)}
\NormalTok{    GE_stn_est_means_df <-}\StringTok{ }\KeywordTok{data.frame}\NormalTok{(}\DataTypeTok{GE_factor=}\KeywordTok{names}\NormalTok{(GE_stn_est_means) , }\DataTypeTok{GE_stn_est=}\NormalTok{GE_stn_est_means)}
    
\NormalTok{    model_dfa <-}\StringTok{ }\KeywordTok{merge}\NormalTok{(model_df2,G_stn_est)}
\NormalTok{    model_dfb <-}\StringTok{ }\KeywordTok{merge}\NormalTok{(model_dfa,E_stn_est)}
\NormalTok{    model_dfc <-}\StringTok{ }\KeywordTok{merge}\NormalTok{(model_dfb, GE_stn_est_means_df)}
    \KeywordTok{head}\NormalTok{(model_dfc)}
\end{Highlighting}
\end{Shaded}

\begin{verbatim}
##   GE_factor E  G phenotype2 phenotype2_stn G_stn_est E_stn_est GE_stn_est
## 1     G10E1 1 10   8.990813       2.201890  1.103228  1.102453   2.204166
## 2     G10E1 1 10   9.014060       2.207582  1.103228  1.102453   2.204166
## 3     G10E1 1 10   9.046218       2.215457  1.103228  1.102453   2.204166
## 4     G10E1 1 10   8.986222       2.200766  1.103228  1.102453   2.204166
## 5     G10E1 1 10   8.963782       2.195271  1.103228  1.102453   2.204166
## 6     G10E1 1 10   9.004545       2.205253  1.103228  1.102453   2.204166
\end{verbatim}

\begin{Shaded}
\begin{Highlighting}[]
\NormalTok{    model_df2 <-}\StringTok{ }\NormalTok{model_dfc}
    
    \KeywordTok{plot}\NormalTok{(model_df2}\OperatorTok{$}\NormalTok{phenotype2_stn, model_df2}\OperatorTok{$}\NormalTok{GE_stn_est)}
    \KeywordTok{abline}\NormalTok{(}\DecValTok{0}\NormalTok{,}\DecValTok{1}\NormalTok{) }\CommentTok{# looks good}
\end{Highlighting}
\end{Shaded}

\includegraphics{20200818_KEL_ParititioningVariance_files/figure-latex/unnamed-chunk-8-2.pdf}

\begin{Shaded}
\begin{Highlighting}[]
\NormalTok{    model_df2}\OperatorTok{$}\NormalTok{mean_stn_est <-}\StringTok{ }\KeywordTok{mean}\NormalTok{(model_df2}\OperatorTok{$}\NormalTok{phenotype2_stn)}
    
    \CommentTok{# Calculate interaction}
\NormalTok{    model_df2}\OperatorTok{$}\NormalTok{int_stn_est <-}\StringTok{ }\NormalTok{model_df2}\OperatorTok{$}\NormalTok{mean_stn_est }\OperatorTok{+}\StringTok{ }\NormalTok{model_df2}\OperatorTok{$}\NormalTok{GE_stn_est }\OperatorTok{-}\StringTok{ }
\StringTok{                        }\NormalTok{model_df2}\OperatorTok{$}\NormalTok{G_stn_est }\OperatorTok{-}\StringTok{   }\NormalTok{model_df2}\OperatorTok{$}\NormalTok{E_stn_est}
\end{Highlighting}
\end{Shaded}

\hypertarget{step-8-compare-true-values-to-estimated-values}{%
\section{Step 8: Compare true values to estimated
values}\label{step-8-compare-true-values-to-estimated-values}}

\begin{Shaded}
\begin{Highlighting}[]
\KeywordTok{head}\NormalTok{(model_df)}
\end{Highlighting}
\end{Shaded}

\begin{verbatim}
##   GE_factor E  G int GE_true G_true E_true mean_true int_true GE_stn_true
## 1     G10E1 1 10   0       9    4.5    4.5         0        0    2.204541
## 2     G10E1 1 10   0       9    4.5    4.5         0        0    2.204541
## 3     G10E1 1 10   0       9    4.5    4.5         0        0    2.204541
## 4     G10E1 1 10   0       9    4.5    4.5         0        0    2.204541
## 5     G10E1 1 10   0       9    4.5    4.5         0        0    2.204541
## 6     G10E1 1 10   0       9    4.5    4.5         0        0    2.204541
##   G_stn_true E_stn_true mean_stn_true int_stn_true            e phenotype
## 1    1.10227    1.10227             0            0  0.003444012  2.207985
## 2    1.10227    1.10227             0            0  0.011321137  2.215862
## 3    1.10227    1.10227             0            0 -0.002250266  2.202291
## 4    1.10227    1.10227             0            0 -0.013903696  2.190637
## 5    1.10227    1.10227             0            0  0.010758599  2.215299
## 6    1.10227    1.10227             0            0 -0.002266453  2.202274
##   phenotype_stn G_stn_est E_stn_est GE_stn_est  mean_stn_est  int_stn_est     I
## 1      2.207582  1.103228  1.102453   2.204166 -7.496512e-18 -0.001514885 FALSE
## 2      2.215457  1.103228  1.102453   2.204166 -7.496512e-18 -0.001514885 FALSE
## 3      2.201890  1.103228  1.102453   2.204166 -7.496512e-18 -0.001514885 FALSE
## 4      2.190240  1.103228  1.102453   2.204166 -7.496512e-18 -0.001514885 FALSE
## 5      2.214895  1.103228  1.102453   2.204166 -7.496512e-18 -0.001514885 FALSE
## 6      2.201874  1.103228  1.102453   2.204166 -7.496512e-18 -0.001514885 FALSE
\end{verbatim}

\begin{Shaded}
\begin{Highlighting}[]
\NormalTok{(true_int <-}\StringTok{ }\KeywordTok{mean}\NormalTok{(}\KeywordTok{abs}\NormalTok{(model_df}\OperatorTok{$}\NormalTok{int_stn_true)))}
\end{Highlighting}
\end{Shaded}

\begin{verbatim}
## [1] 5.440093e-17
\end{verbatim}

\begin{Shaded}
\begin{Highlighting}[]
\CommentTok{# Although we call this "true_int", when int=0 this should be 0}
\CommentTok{# note this is a shortcut that only works with equal sample sizes}

\NormalTok{(obs_int_sim <-}\StringTok{ }\KeywordTok{mean}\NormalTok{(}\KeywordTok{abs}\NormalTok{(model_df}\OperatorTok{$}\NormalTok{int_stn_est)))}
\end{Highlighting}
\end{Shaded}

\begin{verbatim}
## [1] 0.002487851
\end{verbatim}

\begin{Shaded}
\begin{Highlighting}[]
\CommentTok{# when int=0, this will increase as the within-population mean gets less accurate}

\NormalTok{(obs_int_emp <-}\StringTok{ }\KeywordTok{mean}\NormalTok{(}\KeywordTok{abs}\NormalTok{(model_df2}\OperatorTok{$}\NormalTok{int_stn_est)))}
\end{Highlighting}
\end{Shaded}

\begin{verbatim}
## [1] 0.002487851
\end{verbatim}

\hypertarget{drop-the-mic.}{%
\section{Drop the mic.}\label{drop-the-mic.}}


\end{document}
